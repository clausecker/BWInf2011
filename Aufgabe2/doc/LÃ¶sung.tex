\documentclass{scrartcl}

\usepackage[biolinum]{libertine}
\usepackage[euler-digits,small]{eulervm}
\usepackage{amsmath}
\usepackage{amsfonts}
\usepackage{ngerman}
\usepackage{verbatim}

\title{Lösung der Aufgabe~2}
\subtitle{des 30.~Bundeswettbewerbs Informatik}
\author{Robert Clausecker, Alexandra Piloth}
%\date{}

% Syntaxdefinitionen
\newcommand{\ftwo}{\ensuremath{\mathbb F_2}}
\newcommand{\bigO}{O}
\newcommand{\xor}{\oplus}
\begin{document}

\maketitle

\section*{Vorbemerkungen}
Im Folgenden bezeichne $0$ den logischen Wert \emph{falsch} und $1$ den
logischen Wert \emph{wahr}. $a\land b$ ist die logische und-Verknüpfung von $a$
und $b$, $a\xor b$ ist analog die XOR-Verknüpfung. Insbesondere gilt:
$a\xor b=a+b\mod 2$ und $a\land b=a\cdot b\mod 2$.

Die Implementationen der beschriebenen Algorithmen finden sich im Anhang.

\section{Grundidee}
Wir betrachten eine Anordnung von Schaltern und Lampen als \emph{lineares
Gleichungssystem} im Körper \ftwo.  Dieser hat die Elemente $1$ (Lampe an bzw.
Schalter gedrückt) und $0$ (Lampe aus bzw. Schalter nicht gedrückt).

Durch allgemein bekannte Algorithmen, wie das Gaußverfahren, lässt sich eine
Schalterstellung zum Ausschalten aller Lampen in $\bigO(n^3)$ finden; $n$ ist
hierbei die Anzahl der Schalter bzw. Lampen. Nimmt man an, dass ein Bitvektor
der Länge $n$ in ein Maschinenwort passt, so kann eine Lösung sogar in $\bigO(
n^2)$ gefunden werden. Herauszufinden, ob eine Schaltung brauchbar ist, hat eine
ähnliche Komplexität, weil das ein Teilproblem des o.\,g. Problems ist.

\section{Darstellung als Gleichungssystem}
Seien $s_1, s_2,\dots, s_n$ die Zustände der Schalter und $l_1, l_2,\dots,l_n$
die Ausgangszustände der dazu gehörigen Lampen. Jede Lampe wird durch eine Reihe
von Schaltern $s_\alpha, s_\beta,\dots$ umgeschaltet. Diese Umschaltungen lassen
sich als XOR-Operationen darstellen. Der Zustand einer Lampe ist also
\begin{equation}
l_k \xor s_\alpha \xor s_\beta \xor s_\gamma \xor \cdots
\end{equation}
Eine Schalterstellung, welche die Lampe $l_k$ ausschaltet ist damit zugleich
eine Lösung der Gleichung
\begin{equation}
s_\alpha \xor s_\beta \xor s_\gamma \xor \cdots = l_k\label{firstEq}
\end{equation}
Um diese Gleichungen in ein einheitliches Format zu bringen, führen
wir die Koeffizienten $v_{a,b}$ ein. $v_{a,b}$ ist $1$, genau dann wenn Lampe
$l_a$ mit Schalter $s_b$ verbunden ist. Somit lässt sich Gleichung~\ref{firstEq}
als
\begin{equation}
v_{k,1}\land s_1\xor v_{k,2}\land s_2\xor\cdots\xor v_{k,k}\land s_k=l_k
\end{equation}
ausdrücken.

Betrachtet man nun $\xor$ und $\land$ als Addition und Multiplikation im Körper
\ftwo, dann kann man nun die $n$ Gleichungen der Lampen als lineares
Gleichungssystem in den Variablen $s_1$ bis $s_n$ auffassen:
\begin{equation}
\begin{cases}
  v_{1,1}s_1+v_{1,2}s_2+\dots+v_{1,n}s_n&=l_1\\
  v_{2,1}s_1+v_{2,2}s_2+\dots+v_{2,n}s_n&=l_2\\
  \hfil\vdots\hfil&\hfil\vdots\hfil\\
  v_{n,1}s_1+v_{n,2}s_n+\dots+v_{n,n}s_n&=l_n\\
\end{cases}
\end{equation}

Dieses lässt sich wiederum durch die Koeffizientenmatrix
\begin{equation}
M=
\begin{pmatrix}
  v_{1,1}&v_{1,2}&\cdots&v_{1,n}\\
  v_{2,1}&v_{2,2}&\cdots&v_{2,n}\\
  \vdots &\vdots &\ddots&\vdots\\
  v_{1,n}&v_{2,n}&\cdots&v_{n,n}
\end{pmatrix}
\end{equation}
und den Spaltenvektor $A$ der Ausgangszustände $l_1$ bis $l_n$ darstellen. Für
ein effizientes Arbeiten speichern wir jede Zeile von $M$ in ein- oder mehreren
Maschinenwörtern. Da jeder Eintrag in der Matrix nur ein Bit lang ist, kann man
so über eingebaute Bitoperationen ganze Zeilen effizient addieren. Unter der
Annahme, dass eine Zeile komplett in ein Maschinenwort passt, kann man zwei
Zeilen in $\bigO(1)$ mit einem XOR addieren.

\section{Lösbarkeit einer einzelnen Konfiguration}
Wir betrachten nun eine vollständige Konfiguration einer Schaltung, bestehend
aus der Schaltung $M$ und den Ausgangszuständen $A$. In (\ref{exampleGame}) ist
die Konfiguration angegeben, die in der ersten Teilaufgabe zu analysieren ist.
Dabei ist links $M$ und rechts -- durch einen Balken abgeteilt -- $A$ angegeben.
\begin{equation}
M\vert A=\left(
\begin{array}{ccccccc|c}
  1&1&0&0&0&0&1&0\\
  1&1&1&0&0&0&0&0\\
  0&1&1&1&0&0&0&1\\
  0&0&1&1&1&0&0&0\\
  0&0&0&1&1&1&0&1\\
  0&0&0&0&1&1&1&0\\
  1&0&0&0&0&1&1&1 
\end{array}\right)\label{exampleGame}
\end{equation}

Zunächst wenden wir das \emph{Gaußsche Eliminationsverfahren} an, um die
Matrix in Stufenform zu überführen. Damit erhält man einen Überblick über die
Lösbarkeit des Gleichungssystems. Es gilt: Wenn die Determinante, also das
Produkt der Werte in der Hauptdiagonale, ungleich $0$ ist, dann existiert genau
eine Lösung. Wenn nicht, dann existiert entweder keine oder höchstens so viele
verschiedene Lösungen, wie Nullen in der Diagonale vorhanden sind.

Für den ersten Fall kann man mit dem \emph{Gauß-Jordan-Verfahren} die einzelnen
Lösungen berechnen. Im zweiten Fall transformieren wir die Matrix so, dass alle
Zeilen, in denen $v_{k,k} = 0$ ist, komplett leer sind. Dies muss möglich sein,
weil durch die Stufenform in jeder folgenden Zeile mindestens ein Eintrag
weniger als in der überstehenden stehen muss. Weil die Matrix quadratisch ist,
muss für den Fall, dass die Differenz der Anzahl relevanter Einträge zweier
aufeinander folgender Zeilen größer als $1$ ist, mindestens eine Leerzeile 
(Zeile der Form $(0\ 0\ \dots\ 0\,|\,0)$) am Ende stehen. Diese wird nach oben
in die entsprechende Lücke verschoben. Eine derartig transformierte Matrix mit
Determinante $0$ ist in (\ref{echelonZero}) zu sehen.
\begin{equation}
\left(\begin{array}{cccccc|c}
  1&0&0&0&1&0&1\\
  0&0&0&0&0&0&0\\
  0&0&1&1&1&1&1\\
  0&0&0&1&0&1&0\\
  0&0&0&0&0&0&1\\
  0&0&0&0&0&1&0
\end{array}\right)\label{echelonZero}
\end{equation}

Das Gleichungssystem hat nun entweder keine oder $2^n$ Lösungen, wobei $n$ die
Anzahl der Leerzeilen ist. Ersterer Fall tritt genau dann ein, wenn der zur
Zeile $k$ zugehörige Wert in $A$ ungleich $0$ ist, also es eine Zeile der Form
$0=1$ gibt. Beide Fälle sind in (\ref{echelonZero}) gezeigt; der Erste in Zeile
2, der Zweite in Zeile 5. Im zweiten Fall verwendet man das genauso das
Gauß-Jordan-Verfahren mit dem Unterschied, dass im Falle einer Leerzeile der
Wert für $s_k$ beliebig gesetzt werden kann. Für beide Einsetzungen gibt es
jeweils eine Lösung, was im Endeffekt zu oben beschriebener Anzahl führt.

Angewandt auf die Matrix in (\ref{exampleGame}) ergibt sich die Stufenform
\begin{equation}
\left(\begin{array}{ccccccc|c}
  1&1&0&0&0&0&1&0\\
  0&1&1&1&0&0&0&1\\
  0&0&1&0&0&0&1&0\\
  0&0&0&1&1&0&1&0\\
  0&0&0&0&1&1&1&0\\
  0&0&0&0&0&1&1&1\\
  0&0&0&0&0&0&1&0
\end{array}\right).
\end{equation}
Die Determinante ist $1$, somit existiert genau eine Lösung. Nun kann man das
Gauß-Jordan-Verfahren anwenden, um diese zu bestimmen. Es ergibt sich die
reduzierte Stufenform
\begin{equation}
\left(\begin{array}{ccccccc|c}
  1&0&0&0&0&0&0&0\\
  0&1&0&0&0&0&0&0\\
  0&0&1&0&0&0&0&0\\
  0&0&0&1&0&0&0&1\\
  0&0&0&0&1&0&0&1\\
  0&0&0&0&0&1&0&1\\
  0&0&0&0&0&0&1&0
\end{array}\right)
\end{equation}
aus der man direkt die Lösung ablesen kann:
\begin{equation}
s_1 = 0,\quad
s_2 = 0,\quad
s_3 = 0,\quad
s_4 = 1,\quad
s_5 = 1,\quad
s_6 = 1,\quad
s_7 = 0.
\end{equation}

\section{Bestimmung brauchbarer Schaltungen}
Eine Schaltung ist genau dann brauchbar, wenn es für jeden Tupel von
Ausgangszuständen $(l_1,l_2,\dots,l_n)$ genau eine Kombination von
Schalterzuständen $(s_1,s_2,\dots,s_n)$ gibt, sodass diese alle Lampen
ausschaltet. Das dies gilt, lässt sich durch Anwendung des Schubfachprinzipes
zeigen: Es gibt bei $n$ Lampen genau $2^n$ Kombinationen von Lampenzuständen und
ebenso viele Kombinationen von Schalterzuständen. Offensichtlich kann jede
Schalterkombination genau eine Lampenkombination ausschalten. Gäbe es eine
Lampenkombination, die mehr als eine Lösung hat, muss es also zwangsläufig eine
Andere geben, die keine Lösung hat. Dies ist ein Widerspruch zur Annahme, dass
alle Lampenkombinationen lösbar sind.

Es genügt also, nur die Determinanten zu betrachten, weil diese Aufschluss über
die eindeutige Lösbarkeit einer Schaltung geben. Hier am Beispiel der zweiten
Schaltung im Aufgabenblatt. Ihre Koeffizientenmatrix $M$ ist
\begin{equation}
\begin{pmatrix}
  1&1&0&1&0&0&0&0&0\\
  1&1&1&0&1&0&0&0&0\\
  0&1&1&0&0&1&0&0&0\\
  1&0&0&1&1&0&1&0&0\\
  1&0&1&0&1&0&1&0&1\\
  0&0&1&0&1&1&0&0&1\\
  0&0&0&1&0&0&1&1&0\\
  0&0&0&0&1&0&1&1&1\\
  0&0&0&0&0&1&0&1&1
\end{pmatrix}.
\end{equation}
Nach Anwendung des Gaußschen Eliminationsverfahrens ergibt sich die
Stufenmatrix
\begin{equation}
\begin{pmatrix}
  1&1&0&1&0&0&0&0&0\\
  0&1&0&0&1&0&1&0&0\\
  0&0&1&1&0&0&0&0&1\\
  0&0&0&1&1&1&1&0&1\\
  0&0&0&0&1&1&0&1&1\\
  0&0&0&0&0&1&0&1&0\\
  0&0&0&0&0&0&1&1&0\\
  0&0&0&0&0&0&0&1&1\\
  0&0&0&0&0&0&0&0&1
\end{pmatrix},
\end{equation}
deren Determinante $1$ ist. Das Gleichungssystem hat unabhängig von den
Ausgangszuständen genau eine Lösung, damit ist die Schaltung brauchbar.
\end{document}
