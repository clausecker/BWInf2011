\documentclass{scrartcl}

\usepackage[biolinum]{libertine}
\usepackage[euler-digits,small]{eulervm}
\usepackage{amsmath}
\usepackage{amsfonts}
\usepackage{ngerman}
\usepackage{verbatim}

% Klappt leider nicht mit XeTeX....
%\usepackage{epsdice}
%\newcommand{\dice}[1]{\epsdice{#1}}
\newcommand{\dice}[1]{#1}
\newcommand{\SHAone}{\texttt{SHA1}}
\DeclareMathOperator{\hash}{\texttt{HASH}}

\title{Lösung der Aufgabe~1}
\subtitle{des 30.~Bundeswettbewerbs Informatik}
\author{Tobias Bucher, Robert Clausecker, Alexandra Piloth}
\date{25.~September 2011}

\begin{document}

\maketitle

\section{Grundidee}
Es soll für den Spieler verifizierbar sein, dass die künstliche Intelligenz (im
folgenden kurz KI) nicht »schummeln«, also den Zug des Spieler in der selben
Runde wissen, kann.  Zugleich soll aber der Spieler keine Chance haben, den Zug
des Computers zu wissen, bevor dieser ausgeführt wird.  Die Verifizierung soll
außerdem so wenig wie möglich in das Spielgeschehen eingreifen, sodass ein
zügiges Durchspielen trotzdem möglich ist.

Zur Erfüllung dieser Erwartungen verwenden wir ein \emph{Commitment Scheme}.  Es
erlaubt uns die Verifikation durchzuführen, ohne dass dabei eine der Parteien
Informationen über die andere erhält.  Der Aufwand zur Berechnung der für das
Commitment Scheme nötigen kryptographischen Funktionen ist für den menschlichen
Spieler recht hoch, sodass eine Verifikation in jeder Runde nicht in Frage
kommt.  Es reicht aber aus, nur alle $n$ Runden zu verifizieren, um zu zeigen,
dass die KI nicht fair spielt. Wird sie nur einmal erwischt, ist der
Programmierer entlarvt.

\section{Umsetzung}
Zur Implementierung eines Commitment Schemes verwenden wir eine sichere
Hashfunktion $\hash(x)$, welche unabhängig vom Verfahren an sich ist.  Mögliche
Funktionen werden später beschrieben.

Wenn die KI am Zug ist, führt sie folgendes Verfahren durch. Ihr Zug $z$ wird
durch eine Zahl von $0$ bis $2$ dargestellt, wobei $0$ für Schere, $1$ für Stein
und $2$ für Papier steht:
\begin{enumerate}
\item Bestimme eine hinreichend lange natürliche Zufallszahl $n$.
\item Bestimme den Hash $h = \hash(3n+z)$ und gib diesen aus.
\item Erfrage den Zug $Z$ des Spielers,
\item Gib $z$ und $n$ aus. Der Spieler kann nun die Verifikation durchführen,
\item Teile dem Spieler mit, ob er gewonnen, verloren oder unentschieden
  gespielt hat.
\end{enumerate}

Unter der Annahme, dass es nicht in hinreichender Zeit notwendig ist, eine
Kollision für die Hashfunktion zu finden, kann die KI nicht schummeln.  Täte sie
dass, also gäbe sie einen anderen Zug, als ursprünglich berechnet aus, dann
müsste sich der ausgegebene Hash von einem nachträglich berechneten zwangsläufig
unterscheiden. Das Gegenteil ist aufgrund der angenommenen Kollisionssicherheit
von $\hash(x)$ nicht möglich.  Auch ist es unmöglich, aus dem Hash auf den Zug
der KI zu schließen.

Der Spieler kann jederzeit das Spiel kurzzeitig unterbrechen, um eine
Verifikation durchzuführen. Um den Spielfluss nicht unnötig zu stören, ist es
ausreichend, die Verifikation nur ab und zu durchzuführen.  Verifizierte man
aber in einem festen Intervall, so könnte sich die KI darauf einstellen und
immer nur dann schummeln, wenn keine Verifikation erfolgt. Es ist angebracht,
in zufälligen Abständen zu verifizieren. Die KI hat dann keine Möglichkeit der
Verifizierung durch geschicktes Spielen zu entkommen.

Wir halten es für vertretbar, alle zehn bis zwanzig Spiele eine Verifikation
durchzuführen. Dazu kann der Spieler nach jedem Spiel vier Münzen werfen. Zeigen
alle Kopf, führt man eine Verifikation durch. Die Wahrscheinlichkeit dieses
Ereignisses liegt bei $\frac1{16}$, sodass eine Verifikation im Mittel alle 16
Spiele durchgeführt wird. Ist einem das Intervall zu kurz oder zu lang, kann man
auch mehr oder weniger Münzen nehmen. Eine andere Möglichkeit ist es, mit zwei
Würfeln zu würfeln. Zeigen beide Würfel \dice{6}, so wird verifiziert.  Die
Wahrscheinlichkeit hierfür liegt bei $\frac1{36}$, sodass eine Verifikation im
Mittel alle 36 Spiele eintritt.

Wichtig bei allen oben genannten Möglichkeiten ist, dass eine Verifizierung
jederzeit eintreten kann, es also keine Runde gibt, in der garantiert nicht
verifiziert wird. Gäbe es eine solche, könnte der Computer wieder schummeln.

\section{Hashfunktion}
Es wird eine Hashfunktion $\hash(x)$ benötigt, bei der das Finden von
Kollisionen sehr schwer ist; es sollte zumindest mehrere Minuten dauern eine zu
finden. Mehr ist nicht benötigt, weil bereits eine Verzögerung von einer Minute
auffallen würde.

Wir hatten uns erst gedacht, eine Funktion zu finden, die von Hand oder mit
Hilfe eines Taschenrechners berechenbar ist, sodass man zur Verifikation dem
Computer nicht vertrauen muss. Leider haben wir eine solche nicht finden können,
da alle Kandidaten zwar einfach berechenbar und für einen Menschen auch nicht zu
brechen sind, es allerdings für einen Computer kein Problem darstellt, eine
Kollision in weniger als einer Sekunde zu finden.

Wir haben uns dafür entschieden, eine als kryptographisch sicher anerkannte
Funktion wie \SHAone{} zu verwenden. Es ist dann aber notwendig, ein
vertrauenswürdiges Programm zur Berechnung dieser Hashfunktion zu installieren.
Derartige Programme sind quelloffen im Internet verfügbar.

Da \SHAone{} einen String als Eingabe verlangt kann man $3n+z$ z.\,B. in der
Repräsentation zur Basis 16 (Als Hexadezimalzahl) oder 256 (Jede Stelle ein
Byte) an die Hashfunktion übergeben.

\section{Beispielsitzung}

Es sei angenommen, dass $\hash(x) = \mathop{\SHAone}(x_{16})$ ist, wobei
$x_{16}$ die Darstellung von $x$ im Hexadezimalsystem bezeichnet. Es ist z.\,B.
$9876543210_{16} = \texttt{\dq24CB016EA\dq}$. Eine Spiel eines Menschen gegen
die KI auf der Kommandozeile könnte dann unter Verwendung des oben beschriebenem
Commitment Schemes folgendermaßen aussehen:

\begin{verbatim}
Färöer-KI Version 1.0

Schere ist 0
Stein  ist 1
Papier ist 2

01: Hashcode: 00ac0c4bcc5db6cd93caf66afff301aa35963b87
01: Was spielst du? 1
01: Du hast gewonnen!
01: Zufallszahl: 5393799273203848021
01: Meine Wahl:  0
01: Eingabe der Hashfunktion: E08FDF958F39F9FF

Weiterspielen? (Y/N) Y

02: Hashcode: fbf904a8b82d571eb82295923c9d4b5d66f0723a
02: Was spielst du? 2
02: Du hast verloren!
02: Zufallszahl: 42701395392693193
02: Meine Wahl: 2
02: Eingabe der Hashfunktion: 1C71E141F628B5D

Weiterspielen? (Y/N) N

Du hast 1 Spiel(e) gewonnen.
Du hast 1 Spiel(e) verloren.
Du hast 0 Spiel(e) unentschieden gespielt.
Damit steht es unentschieden.
\end{verbatim}
\end{document}
